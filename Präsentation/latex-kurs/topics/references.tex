\subsection{Querverweise}\label{sec:querverweise}

\begin{frame}[c,fragile]
\frametitle{Querverweise}

Es gibt in \LaTeX\ drei Befehle zur Benutzung von Querverweisen:

\begin{minipage}{0.45\textwidth}
\begin{block}{Querverweise}
\begin{verbatim}
\label{marker}
\ref{marker}
\pageref{marker}
\end{verbatim}
\end{block}
\end{minipage}
\hfill
\begin{minipage}{0.45\textwidth}
\begin{block}{Pr\"afixe}
\begin{tabular}{ll}
\texttt{chap:} & chapter \\
\texttt{sec:} & section \\
\texttt{fig:} & figure \\
\texttt{tab:} & table \\
\texttt{eq:} & equation \\
\texttt{lst:} & listing
\end{tabular}
\end{block}
\end{minipage}

Der \texttt{marker} ist frei w\"ahlbar. Es ist jedoch \"ublich einen Präfix zu benutzen, der angibt was referenziert wird:

\verb|\label{sec:Querverweise}|

\end{frame}

\begin{frame}[c,fragile]
\frametitle{Definition von Markern}

F\"ur \texttt{figure}-- und \texttt{table}--Umgebungen ist wichtig, dass das Label innerhalb des \lc{caption}--Befehls angegeben wird.

\begin{block}{Beispiel}
\begin{verbatim}
\section{Querverweise}\label{sec:querverweise}

\begin{figure}
\caption{Sch\"ones Bild\label{fig:beispielfig}}
\end{figure}

\begin{table}
\caption{Wichtige Tabelle\label{tab:beispieltab}}
\end{table}

\begin{equation}\label{eq:fibonacci}
\end{equation}

\begin{defi}[Taylorpolynom]\label{th:taylorpolynom}
\end{defi}
\end{verbatim}

\end{block}

\end{frame}

\begin{frame}[c,fragile]
\frametitle{Referenzieren von Markern}

\begin{block}{Beispiele}
\begin{verbatim}
siehe Seite~\pagref{sec:querverweise}
siehe Abbildung~\ref{fig:beispielfig}
siehe Tabelle~\ref{tab:beispieltab}
siehe Gleichung~\ref{eq:fibonacci}
  auf Seite~\pageref{eq:fibonacci}
in Definition~(\ref{th:taylorpolynom})
\ref{unbekannt}, \pageref{unbekannt}
\end{verbatim}
\end{block}

\begin{block}{Ausgabe}
siehe Seite~\pageref{sec:querverweise} \\
siehe Abbildung~\ref{fig:beispielfig} \\
siehe Tabelle~\ref{tab:beispieltab} \\
siehe Gleichung~\ref{eq:fibonacci} auf Seite~\pageref{eq:fibonacci} \\
in Definition~(\ref{th:taylorpolynom}) \\
\ref{unbekannt}, \pageref{unbekannt}
\end{block}

\end{frame}

