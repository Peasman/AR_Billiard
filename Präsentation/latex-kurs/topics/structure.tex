\section{Aufbau von \LaTeX--Dokumenten}

\begin{frame}[c,fragile]
\frametitle{Aufbau von \LaTeX--Dokumenten}
\begin{block}{Minimales \LaTeX--Dokument}
\begin{verbatim}
% Präambel
\documentclass{article}
\begin{document}
% ... Text ...
\end{document}
\end{verbatim}
\end{block}

\begin{itemize}
\item Kommentare beginnen mit dem Prozentzeichen \texttt{\%}
\item \LaTeX-\emph{Befehle} beginnen mit einem \emph{backslash} \lc{befehl}.
\item \emph{Umgebungen} sind Textbl\"ocke, die mit \lcp{begin}{\textit{umgebung}}, \lcp{end}{\textit{umgebung}} umschlossen sind.
\item Geschweifte Klammern \texttt{\{}, \texttt{\}} fassen Text zu Bl\"ocken zusammen.
\item Befehle und Umgebungen k\"onnen Parameter verarbeiten: \lc{emph\{Betonter Text\}} ergibt \emph{Betonter Text}. 
\item Optionale Parameter k\"onnen erscheinen in eckigen Klammern zu Anfang der Parameterliste: \\
  \lcopp{befehl}{optional}{parameter1}{parameter2}
\end{itemize}
\end{frame}

\subsection{Dokumentenklassen}

\begin{frame}[c,fragile]
\frametitle{Dokumentenklassen}

\begin{block}{}
\lcop{documentclass}{option1,option2}{klasse} \\
\lcop{documentclass}{11pt,a4paper}{article}
\end{block}

\LaTeX-Dokumentenklassen:
\begin{description}
\item[\texttt{article}] Artikel, Vortr\"age, Ausarbeitungen, Berichte
\item[\texttt{scrartcl}] Deutsche, auf DinA4 angepasste Version
\item[\texttt{report}] L\"angere Berichte, Diplomarbeiten, Dissertationen, Skripte
\item[\texttt{scrreprt}] Deutsche, auf DinA4 angepasste Version
\item[\texttt{book}] B\"ucher
\item[\texttt{scrbook}] Deutsche, auf DinA4 angepasste Version
\item[\texttt{letter}] Briefe
\item[\texttt{scrlttr2}] Briefe nach DIN 676
\end{description}
\end{frame}

\begin{frame}[c]
\frametitle{G\"angige Dokumentenoptionen}

H\"aufig benutzte Dokumentenoptionen:
\begin{description}
\item[\texttt{draft}] Entwurfsmodus (markiert überstehenden Text, zeigt Grafiken als Box)
\item[\texttt{11pt}] Schriftgr\"o\ss{}e 11 Punkte
\item[\texttt{12pt}] Schriftgr\"o\ss{}e 12 Punkte
\item[\texttt{a4paper}] Gr\"o\ss{}en anpassen f\"ur Din A4
\item[\texttt{titlepage}] Titelseite erzwingen f\"ur \emph{\texttt{article}}
\item[\texttt{twocolumn}] Zweispaltiges Dokument
\item[\texttt{twoside}] Vorder- und R\"uckseite der Seiten wird bedruckt
\item[\texttt{fleqn}] Formeln linksb\"undig setzen
\item[\texttt{leqno}] Nummerierung von Formeln links statt rechts
\end{description}
\end{frame}

\begin{frame}[c]
\frametitle{Einbindung von Paketen}

Mit dem Befehl \lc{usepackage} k\"onnen zus\"atzliche Makropakete aktiviert werden.
\begin{itemize}
\item \lcp{usepackage}{amssymb,amsmath} Spezielle Mathematische Symbole und Umgebungen
\item \lcp{usepackage}{graphicx} Einbindung von Grafiken
\item \lcp{usepackage}{algorithm} Darstellung von Algorithmen
\item \lcp{usepackage}{hyperref} Umgang mit Hyperlinks
\item \lcp{usepackage}{fancyhdr} Selbstdefiniert Kopf- und Fu\ss{}zeilen
\item \lcop{usepackage}{bottom=2cm}{geometry} Seitengeometrie \"andern
\end{itemize}
\end{frame}

\subsection{Eingabe von Text}

\begin{frame}[c,fragile]
\frametitle{Eingabe von Text}

\begin{itemize}
\item F\"ur die Eingabe von Text steht der 7bit-ASCII-Zeichensatz zur Verf\"ugung $\Longrightarrow$ keine Sonderzeichen
\item Manche Zeichen haben f\"ur \TeX{} eine spezielle Bedeutung: z.B.{} \verb+\+, \verb+&+, \verb+$+
\item \glqq Escape\grqq-Zeichen: \verb+\+
\end{itemize}

\begin{block}{}
\begin{center}

\begin{tabular}{ll|ll}
\multicolumn{2}{c}{\textbf{Sonderzeichen}} & \multicolumn{2}{c}{\textbf{Akzente}} \\ \hline
\verb+$\backslash$+ & $\backslash$ & \verb+\^{a}+ & \^{a} \\
\verb|\textbackslash| & \textbackslash & \verb+\`{a}+  & \`{a} \\
\verb+\$+ & \$ & \verb+\'{a}+  & \'{a} \\
\verb+\%+ & \% & \verb+\c{c}+  & \c{c} \\
\verb+\{+ & \{ & \verb+\~{a}+  & \~{a} \\
\verb+\}+ & \} & \verb+\"{i}+  & \"{i} \\
\verb+\#+ & \# & \verb+\o+     & \o \\
\verb+\_+ & \_ \\
\verb+\&+ & \& &
\end{tabular}

\end{center}
\end{block}

\end{frame}

\begin{frame}[c,fragile]
\frametitle{Eingabe deutscher Texte}

Zur Eingabe deutscher Texte:
\begin{itemize}
\item \lcop{usepackage}{german}{babel}
  \begin{itemize}
  \item Deutsche Überschriften \\ \glqq Inhaltsverzeichnis\grqq{} statt \glqq Table of contents\grqq
  \item K\"urzel f\"ur deutsche Umlaute: \\ \verb+\"a\"A\"o\"O\"u\"U\ss{}+ $\Longrightarrow$ \"a\"A\"o\"O\"u\"U\ss{}
  \item K\"urzel f\"ur deutsche Umlaute: \\ \verb+"a"A"o"O"u"U"s+ $\Longrightarrow$ "a"A"o"O"u"U"s
  \item \verb+Fu\ss{}pilz+ $\Longrightarrow$ Fu\ss{}pilz, \verb+Ma\ss{} halten+ $\Longrightarrow$ Ma\ss{} halten
  \end{itemize}
\item \lcop{usepackage}{ngerman}{babel}
  \begin{itemize}
  \item Sprachpaket f\"ur die neue deutsche Rechtschreibung
  \end{itemize}
\item \lcop{usepackage}{latin1}{inputenc} f\"ur Unix (ISO-8859-1)
\item \verb|\usepackage[ansinew]{inputenc}| für Windows
\item \verb|\usepackage[applemac]{inputenc}| für den Mac
\item \lcop{usepackage}{utf8}{inputenc} UTF-8-Kodierung
\end{itemize}

\end{frame}

\begin{frame}[c,fragile]
\frametitle{Silbentrennung}

\LaTeX{} hat ein eingebautes W\"orterbuch und einen Regelsatz nach dem es Worte trennt. Da dies jedoch f\"ur ungebr\"auchliche
W\"orter nicht immer klappt, gibt es die M\"oglichkeit, die Trennstellen vorzugeben:

\begin{block}{Trennstellen}
\verb|Donau\-dampf\-schiff\-fahrts\-gesell\-schaft| \\
\verb|Karl-""Heinz-""Stra\ss e| \\
\verb|\hyphenation{Donau-dampf-schiff-fahrts-gesell-schaft}|
\end{block}

Der \lc{hyphenation}--Befehl kann in der Pr\"aambel angegeben werden und gibt die Schreibung f\"ur das ganze Dokument vor.

Mit dem Befehl \lc{sloppy} in der Pr\"aambel kann festgelegt werden, dass im Dokument nie getrennt wird.


\end{frame}

\begin{frame}[c,fragile]
\frametitle{Anf\"uhrungszeichen}

\begin{block}{Anf\"uhrungszeichen}
Auf dem Schild steht \glq Eingang\grq. \\
Sie sagte, \glqq Das Vergn\"ugen ist ganz meinerseits.\grqq \\
\flq Parlez-vous fran\c{c}ais?\frq \\
\flqq Sur le pont d'Avignon\ldots\frqq \\
I dit it `my way'. \\
He's singing ``Yippie ya ya yippie yeah''.
\end{block}

\begin{itemize}
\item Einfache deutsche Anf\"uhrungszeichen: \verb|\glq|, \verb|\grq| \\
  \textbf{g}erman \textbf{l}eft/\textbf{r}ight \textbf{q}uote
\item Doppelte deutsche Anf\"uhrungszeichen: \verb|\glqq|, \verb|\grqq| \\
  \textbf{qq} $=$ double quote
\item Einfache franz\"osische Anf\"uhrungszeichen: \verb|\flq|, \verb|\frq| \\
  \textbf{f}rench
\item Doppelte franz\"osische Anf\"uhrungszeichen: \verb|\flqq|, \verb|\frqq|
\item Einfache englische Anf\"uhrungszeichen: \verb|`|, \verb|'|
\item Doppelte englische Anf\"uhrungszeichen: \verb|``|, \verb|''|
\end{itemize}
\end{frame}

\begin{frame}[c,fragile]
\frametitle{Typographische Besonderheiten}
\begin{block}{Gedankenstriche}
\begin{description}
\item[\texttt{-}] Trennungsstrich zwischen zwei W\"ortern bei Silben-\\trennung
\item[\texttt{-{}-}] Gedankenstrich im deutschen oder Bindestrich: \LaTeX--Kurs
\item[\texttt{-{}-{}-}] Nur im englischen: ---Gedankenstrich---
\end{description}
\end{block}

\begin{block}{Auslassungspunkte und Leerzeichen}
\begin{center}
\nonfrenchspacing
\begin{tabular}{lll}
\dots & \lc{dots} & Auslassungspunkte \\
... & \texttt{...} \\
Nr. 1 & \verb|Nr. 1| & einfaches Leerzeichen \\
Nr.\ 1 & \verb|Nr.\ 1| & maskiertes Leerzeichen \\
Nr.~1 & \verb|Nr.~1| & gesch\"utztes Leerzeichen
\end{tabular}
\end{center}
\end{block}

\begin{block}{\lc{(non)frenchspacing}}
\begin{center}
\begin{tabular}{ll}
{\frenchspacing Satzende. Satzanfang} &
\verb|{\frenchspacing Satzende. Satzanfang}| \\
{\nonfrenchspacing Satzende. Satzanfang} &
\verb|{\nonfrenchspacing Satzende. Satzanfang}|
\end{tabular}
\end{center}
\end{block}

\end{frame}

\section{Strukturierung des Dokuments}
\subsection{Titelseite}

\begin{frame}[c,fragile]
\frametitle{Titelseite}

\begin{block}{}
\begin{verbatim}
\documentclass[a4paper]{report}
\begin{document}
\title{Nicht wichtig}
\author{Hans~Dampf \and Lore~Lei \and Polly~Morph}
\date{\today}
\maketitle
\end{document}
\end{verbatim}
\end{block}

\begin{description}
\item[\texttt{\~}] Unzerbrechliches Leerzeichen
\item[\lc{title}] Titel des Dokuments
\item[\lc{author}] Autoren (Schl\"usselwort \lc{and} bei Angabe mehrerer Autoren)
\item[\lc{date}] Ver\"offentlichungsdatum
\item[\lc{maketitle}] Generieren der Titelseite aus den Metadaten
\end{description}

\end{frame}

\begin{frame}[c,fragile]
\frametitle{Titelseite selbst gestalten}

Um die Titelseite komplett selbst zu gestalten benutzt man die \texttt{titlepage}-Umgebung:
\begin{block}{Beispiel}
\begin{verbatim}
\begin{titlepage}
\textbf{\large{Titeltext}}

geschrieben von mir\ldots
\end{titlepage}
\end{verbatim}
\end{block}

\begin{block}{Ausgabe}
\textbf{\large{Titeltext}}

geschrieben von mir\ldots
\end{block}

Die Titelseite taucht an der Stelle im Text auf, an der sie definiert wurde. Der Befehl \lc{maketitle} wird dann nicht verwendet.

Nur bei der Buch--Dokumentenklasse nimmt die Titelseite auch tats\"achlich eine ganze Seite ein.

\end{frame}

\begin{frame}[c,fragile]
\frametitle{Zusammenfassung}

Einf\"ugen einer Zusammenfassung des Inhalts. Die Dokumentenklasse hat Auswirkungen auf die Darstellung.

\begin{block}{}
\begin{verbatim}
\begin{abstract}
... Zusammenfassung des Textes ...
\end{abstract}
\end{verbatim}
\end{block}

Zur \"Anderung des angezeigten Namens der Zusammenfassung nutzt man den Befehl
\begin{verbatim}\renewcommand{\abstractname}{\"Uberblick}\end{verbatim}

\end{frame}

\begin{frame}[c,fragile]
\frametitle{Strukturierung des Dokuments}

Gliederung des Dokuments in Kapitel und Unterkapitel:
\begin{itemize}
\item \lcp{part}{}
\item \lcp{chapter}{} \emph{nur in der Buch- oder Report-Dokumentenklasse}
\item \lcp{section}{}
\item \lcp{subsection}{}
\item \lcp{subsubsection}{}
\item \lcp{paragraph}{}
\item \lcp{subparagraph}{}
\end{itemize}

Als optionaler Parameter kann eine Kapitelbezeichnung angegeben werden, die nur im Inhaltsverzeichnis auftaucht.

Anh\"angen von \verb|*| an den Kapitelbefehl verhindert, dass das Kapitel nummeriert und im Inhaltsverzeichnis aufgef\"uhrt wird:
\verb|\section*{...}|

Mit dem Befehl \lc{appendix} kann markiert werden, dass die nachfolgenden Kapitel zum Anhang gehören.
\end{frame}

\begin{frame}[c,fragile]
\frametitle{Mehrere Quelldateien}

Der \lc{input}--Befehl wird genutzt, um kleine Textst\"ucke in den laufenden Text einzuf\"ugen.
\begin{block}{Einbinden von Textst\"ucken}
\begin{verbatim}...wie in folgendem Programm:
\input{cplusplus_beispiel}
\end{verbatim}
\end{block}


Der \lc{include}--Befehl wird zur Gliederung des \LaTeX--Quellcodes benutzt: einzelne Kapitel werden in eigene Dateien geschrieben
und in der Hauptdatei mit \lc{include} in das Gesamtdokument eingebunden.

\begin{block}{Einbinden weiterer Dateien}
\begin{verbatim}\includeonly{kapitel2,kapitel3}
\include{kapitel1}
%========================================================================================
% TU Dortmund, Informatik Lehrstuhl VII
%========================================================================================


%========================================================================================
% TU Dortmund, Informatik Lehrstuhl VII
%========================================================================================

\chapter{Ergebnisse}


\section{Aufbau der Umfrage}
\section{Darstellung der Ergebnisse}
\subsection{Benutzerfreundlichkeit}
\subsection{Genauigkeit der Spielsteuerung}



\end{verbatim}
\end{block}

\lc{includeonly} bindet nur die angegebenen Dateien ein. Auf diese Weise muss nicht immer das vollst\"andige Dokument kompiliert werden.

\end{frame}

\begin{frame}[c]
\frametitle{Umbr\"uche}
\justifying

\textbf{Abs\"atze:} Leerzeilen erzeugen neue Abs\"atze, \\[0.2mm]

alternativ kann \lc{par} benutzt werden.\par

\textbf{Zeilenumbruch:} \texttt{\textbackslash\textbackslash} \\ oder \lc{newline} \\ bricht eine Zeile um. \lc{linebreak} \linebreak gleicht den Wortabstand im Blocksatz aus.

\textbf{Seitenwechsel:} \lc{newpage} erzwingt einen Seitenwechsel, \lc{pagebreak} versucht die Seite durch Ausgleich des Abstands zwischen den Abs\"atzen auszuf\"ullen.
Die Befehle \lc{clearpage} und \lc{cleardoublepage} beenden ebenfalls die Seite und f\"ugen nachfolgend alle Gleitobjekte ein, die noch nicht ausgegeben wurden. Gleitobjekte sind u.a.{} Abbildungen und Tabellen.

Bei den \texttt{break}-Befehlen kann als optionaler Parameter angegeben werden, mit welcher Priorit\"at (0--4) ein Seitenumbruch stattfinden soll.

Die \texttt{no\ldots{}break}-Befehle verhindern den Umbruch.

\end{frame}

\begin{frame}[c,fragile]
\frametitle{Aufz\"ahlungen}

Drei Arten von Aufzählungen:
\begin{description}
\item[\texttt{itemize}] Einfache Aufz\"ahlung
\item[\texttt{enumerate}] Nummerierte Aufz\"ahlung
\item[\texttt{description}] Beschreibung
\end{description}

\begin{minipage}{0.3\textwidth}
\begin{block}{\texttt{itemize}}
\begin{itemize}
\item A
\item B
\item[$\circ$] C
\end{itemize}
\begin{verbatim}
\begin{itemize}
\item A
\item B
\item[$\circ$] C
\end{itemize}
\end{verbatim}
\end{block}
\end{minipage}
\hfill
\begin{minipage}{0.3\textwidth}
\begin{block}{\texttt{enumerate}}
\begin{enumerate}
\item A
\item B
\item C
\end{enumerate}
\begin{verbatim}
\begin{enumerate}
\item A
\item B
\item C
\end{enumerate}
\end{verbatim}
\end{block}
\end{minipage}
\hfill
\begin{minipage}{0.3\textwidth}
\begin{block}{\texttt{description}}
\begin{description}
\item[A] \ldots
\item[B] \ldots
\item[C] \ldots
\end{description}
\begin{verbatim}
\begin{description}
\item[A] \ldots
\item[B] \ldots
\item[C] \ldots
\end{description}
\end{verbatim}
\end{block}
\end{minipage}

\end{frame}

\begin{frame}[c,fragile,squeeze]
\frametitle{Verschachtelte Aufz\"ahlungen}

\begin{minipage}[t]{0.48\textwidth}
\begin{block}{Beispiel}
\begin{verbatim}
\begin{description}
\item[Zutaten]
  \begin{itemize}
    \item 200g Mehl
    \item 5 Eier
    % ...
  \end{itemize}
\item[Zubereitung]
  \begin{enumerate}
    \item Mehl und Milch in
          eine Sch\"ussel
    \item Glatt r\"uhren
    % ...
  \end{enumerate}
\end{description}
\end{verbatim}
\end{block}
\end{minipage}
\hfill
\begin{minipage}[t]{0.48\textwidth}
\begin{block}{Beispiel}
\begin{description}
\item[Zutaten]
  \begin{itemize}
    \item 200g Mehl
    \item 5 Eier
    \item 400ml Milch
    \item 1 Prise Salz
    \item 25g Zucker
  \end{itemize}
\item[Zubereitung]
  \begin{enumerate}
    \item Mehl und Milch in
          eine Sch\"ussel
    \item Glatt r\"uhren
    \item Eier, Salz und Zucker unterr\"uhren
    \item Mit ein wenig Butter Pfannkuchen backen
  \end{enumerate}
\end{description}
\end{block}
\end{minipage}

\end{frame}

\subsection{Tabulatoren}

\begin{frame}[c,fragile]
\frametitle{Tabulatoren}

\begin{block}{Beispiel}
\begin{verbatim}
\begin{tabbing}
erste Spalte \= zweite Spalte \= dritte Spalte\kill
erstes \> zweites \> drittes\\
vorne  \> mitte   \> hinten \+ \\
mitte  \> hinten  \+ \\
hinten \- \- \\
vorne \> mitte \> hinten \\
\end{tabbing}
\end{verbatim}
\end{block}

\begin{block}{Ausgabe}
\begin{tabbing}
erste Spalte \= zweite Spalte \= dritte Spalte\kill
erstes \> zweites \> drittes\\
vorne  \> mitte   \> hinten \+ \\
mitte  \> hinten  \+ \\
hinten \- \- \\
vorne \> mitte \> hinten \\
\end{tabbing}
\end{block}

\end{frame}

\begin{frame}[c,fragile]
\frametitle{Erkl\"arung}

\begin{itemize}
\item \begin{block}{}\verb|erste Spalte \= zweite Spalte \= dritte Spalte\kill|\end{block} 
Vorlage f\"ur die Zeile \lc{=} legt die Tabpositionen fest. \\
\verb|\kill| l\"oscht die Zeile.
\item \begin{block}{}\verb|erstes \> zweites \> drittes\\|\end{block}
\verb|\>| erzeugt den Sprung auf die n\"achste Tabposition.
\item \begin{block}{}\verb|vorne  \> mitte   \> hinten \+|\end{block}
\verb|\+| r\"uckt alle folgenden Zeilen um eine Tabposition ein.
\item \begin{block}{}\verb|hinten \- \- \\|\end{block}
\verb|\-| hebt eine Einr\"uckungsstufe f\"ur alle folgenden Zeilen auf.
\item \verb|\pushtabs| speichert die aktuelle Tabulatorkonfiguration ab.
\item \verb|\poptabs| stellt die letzte gespeicherte Tabulatorkonfigration wieder her.
\end{itemize}

\end{frame}

\subsection{Tabellen}

\begin{frame}[c,fragile]
\frametitle{Tabellen}

\begin{block}{Beispiel}
\begin{verbatim}
\begin{tabular}{c|c||c}
$x$ & $y$ & Summe \\ \hline
1 & 1 & 2 \\
1 & 2 & 3 \\
2 & 3 & 5 
\end{tabular}
\end{verbatim}
\end{block}

\begin{block}{Ausgabe}
\begin{tabular}{c|c||c}
$x$ & $y$ & Summe \\ \hline
1 & 1 & 2 \\
1 & 2 & 3 \\
2 & 3 & 5 
\end{tabular}
\end{block}

\end{frame}

\begin{frame}[c,fragile]
\frametitle{Tabellenformat}

%\begin{minipage}{0.45\textwidth}
\begin{block}{Beispiel}
\begin{verbatim}
\begin{tabular}{l|c|r|p{3cm}||r@{$\times$}l}
links & zentriert & rechts & $3$ cm & 8 & 3 \\
\end{tabular}
\end{verbatim}
\end{block}

\begin{block}{Ausgabe}
\begin{tabular}{l|c|r|p{3cm}||r@{$\times$}l}
links & zentriert & rechts & $3$ cm & 8 & 3 
\end{tabular}
\end{block}
%\end{minipage}

Tabellenformat:
\begin{description}
\item[\texttt{l}] Zelleninhalt linksb\"undig ausrichten
\item[\texttt{c}] Zelleninhalt zentriert ausrichten
\item[\texttt{r}] Zelleninhalt rechtsb\"undig ausrichten
\item[\texttt{p\{$n$\}}] Zelle mit fester Breite $n$
\item[\texttt{|}] Einfacher Trennstrich zur n\"achsten Zelle
\item[\texttt{||}] Doppelter Trennstrich zur n\"achsten Zelle
\item[\texttt{@\{text\}}] Benutzerdefiniertes Trennzeichen
\end{description}

\end{frame}

\begin{frame}[c,fragile]
\frametitle{Tabelleninhalt}

\begin{minipage}{0.55\textwidth}
\begin{block}{Beispiel}
\begin{verbatim}
eins & zwei & drei \\ \hline
vier \vline{} f\"unf & sechs & \\
\cline{1-2}
\multicolumn{3}{|c|}{sieben} \\
acht & neun & zehn \\
\end{verbatim}
\end{block}
\end{minipage}
\hspace*{1ex}
\begin{minipage}{0.4\textwidth}
\begin{block}{Ausgabe}
\begin{tabular}{lll}
eins & zwei & drei \\ \hline
vier \vline{} f\"unf & sechs & \\
\cline{1-2}
\multicolumn{3}{|c|}{sieben} \\
acht & neun & zehn 
\end{tabular}
\end{block}
\end{minipage}

\begin{description}
\item[\texttt{\&}] Trennzeichen f\"ur Spalten
\item[$\mathtt{\backslash\backslash}$] Trennzeichen f\"ur Zeilen
\item[\lc{hline}] Linie \"uber die ganze Breite
\item[\lc{vline}] Vertikale Linie innerhalb einer Spalte
\item[\lcp{cline}{m-n}] Linie von Spalte $m$ bis Spalte $n$
\item[\lcppp{multiline}{n}{format}{Inhalt}] Spalte \"uber $n$ Spalten strecken mit Format \texttt{format} und Inhalt \texttt{Inhalt}
\end{description}

\end{frame}

