%========================================================================================
% TU Dortmund, Informatik Lehrstuhl VII
%========================================================================================

\chapter{Problemstellung}
Rendering und Phsyik \\
Das Erstellen des Spielfeldes an sich, stell dabei die erste Herausforderung dar: Man benötigt ein Spielfeld bzw. einen virtuellen Billardpool, bestehend aus Platte und Löchern. 
Außerdem braucht man Kugeln, die entweder Halb oder voll sind, verschiedene Nummern tragen, als auch verschieden Farben haben. 
Die Kugeln müssen sich zudem auch noch bewegen können und miteinander, als auch mit der Wand und den Löchern kollidieren können. 
Kamera Kalibrierung \\
Kö-Detektion \\
GUI-Zusammenbau \\
Die Graphische Oberfläche(GUI=Graphical User Interface) ist das Endprodukt unseres Projektes. Hierbei muss man die Funktionen und alle Arten von Komponenten die in Rendering und Physik, Kamera Kalibrierung sowie die Queue-Erkennung sind, zusammenfügen und ein Laufendes funktionieredes Spielfeld-Fenster erstellen. Die Herausforderung hierbei ist das man genau drauf achten muss, was die einzelen funktionen tun und die passenden Elemente in der Qt zu finden, die die funktionen der anderen Klassen unterstützen könnten.
Ein besondere Herausforderung besteht jedoch dabei, denn Benutzern zu Informieren, welcher Spieler welchen Balltypen bekommt und wer gerade am Zug ist.

