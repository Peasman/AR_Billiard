%========================================================================================
% TU Dortmund, Informatik Lehrstuhl VII
%========================================================================================

\chapter{Problemstellung}
\textbf{Rendering und Phsyik} \\
Das Erstellen des Spielfeldes an sich, stell dabei die erste Herausforderung dar: Man benötigt ein Spielfeld bzw. einen virtuellen Billardpool, bestehend aus Platte und Löchern. Außerdem braucht man Kugeln, die entweder Halb oder voll sind, verschiedene Nummern tragen, als auch verschieden Farben haben. Die Kugeln müssen sich zudem auch noch bewegen können und miteinander, als auch mit der Wand und den Löchern kollidieren können. 
\\\\\textbf{Kamera Kalibrierung}\\
Eine weitere Herausforderung ist die korrekte Erkennung des Spielfeldes im Kamerakoordinatensystem um damit das Transformieren eines gegebenen Punktes aus der Kamera ins Spielfeldkoordinatensystem zu gewährleisten. Darunter fällt das Rendern eines bekannten Patternmusters, welches von der Kamera erkannt werden kann, sowie die anschließende Entzerrung des Kamerabildes um das 3D-Kamerakoordinatensystem in das 2D-Spielfeldkoordinatensystem zu transformieren.
\\\\\textbf{Kö-Detektion} \\
blabla
\\\\\textbf{GUI-Zusammenbau} \\
blabla
