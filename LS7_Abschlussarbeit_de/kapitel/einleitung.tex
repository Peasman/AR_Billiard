%========================================================================================
% TU Dortmund, Informatik Lehrstuhl VII
%========================================================================================

\chapter{Einleitung}

\section{Motivation und Hintergrund}
Die Aufgabe des Fachprojekts war eine Anwendung zu schreiben, die ein nicht-triviales Eingabegerät benutzt, genauso wie eine nicht-triviale, graphische Ausgabe. Wir haben uns dafür entschieden als Eingabe eine Kamera in Kombination mit OpenCV zu nutzen und als Ausgabe Rendering per OpenGL. Da wir bereits im Vorraus ein Projekt hatten, was ähnlich wie Billiard funktioniert hat, nämlich ein Airhockey Spiel auf einem Tablet-Tisch, haben wir uns dafür entschieden das als Basis zu benutzen, um unsere Probleme hauptsächlich auf die beiden geforderten Gebiete zu verschieben. Damit hatten wir als Grundlage ein bereits funktionierendes Kollisionssystem für eine Kugel mit statischen Objekten, sowie leichtes Rendering für Kreise. Zudem gab es auch eine Vorbereitungsaufgabe in OpenCV die uns schon ein Verständnis für Kamera-Kalibrierung gebracht hat. 
\section{Zielsetzung}
Es soll ein Billiard-Computerspiel entstehen, welches durch einen Beamer auf einen Tisch projiziert wird und dann mit einem gewöhnlichen, schwarz gefärbten Stock gespielt wird. Der Stock fungiert dabei als Queue und wird mit einer Kamera erkannt. 
\section{Aufbau der Arbeit}
Wir werden zunächst unsere Probleme erläutern, die uns auf dem Weg zum Endprodukt aufgekommen sind. Anschließend werden wir erklären, wie wir diese Probleme gelöst haben. Zum Schluss wird dann das Endprodukt evaluiert und unsere angewandten Lösungsstrategien diskutiert.

