%========================================================================================
% TU Dortmund, Informatik Lehrstuhl VII
%========================================================================================

\chapter{Ergebnisse}


\section{Aufbau der Umfrage}
Unsere Umfrage ist während der Präsentation unseres Projektes durchgeführt worden. Dafür waren alle anderen Gruppen des Fachprojektes anwesend, die zum Teil mit ähnlicher Technologien gearbeitet haben. Den Personen, die unser Spiel ausprobiert haben, wurde ein Fragebogen gegeben, mit dem sie unser Programm bewerten sollten. Der Fragebogen sah wie folgt aus: 
\begin{itemize}
	\item [1.] Wie viel Spaß macht das Spiel?
	\item [2.] Wie genau ist die Steuerung?
	\item [3.] Wie realistisch sind die Physikberechnungen unabhängig vom Queue?
\end{itemize}
Für jede Frage konnte man Punkte von 0 bis 5 vergeben, wobei 0 das Schlechteste  und 5 das Beste Ergebnis ist.
\section{Darstellung der Ergebnisse}
Insgesamt wurden während der Präsentation von 5 unterschiedlichen Personen Bewertungen zu den einzelnen Fragestellungen abgegeben. Die Ergebnisse sind in folgender Tabelle zu sehen:

\begin{table}[H]
\centering
\begin{tabular}{|l | c c c c c|r|}
	\hline
	\diagbox{Kriterium}{Person} & 1 & 2 & 3 & 4 & 5 & Durchschnitt\\	
	\hline 
	Spaß & 5 & 3 & 3 & 4 & 4 & 3,8\\
	Genauigkeit & 2 & 1 & 1 & 3 & 2 & 1,8\\
	Physik & 4 & 3 & 4 & 3 & 4 & 3,6\\
	\hline

\end{tabular}
	\caption{Ergebnisse der Umfrage}
\end{table}

Während der Spaß und die Physiksimulation mit Noten zwischen 3 und 4 bewertet wurde, wurde die Genauigkeit der Spielsteuerung eher negativ empfunden. 
Ein möglicher Grund für diese Bewertung könnte im Testaufbau liegen, da dieser nicht auf einem Tisch sondern auf einer Leinwand war.
Gerade in dem Bereich der Erkennung und Kalibrierung liegt daher noch ein großes Verbesserungspotenzial.