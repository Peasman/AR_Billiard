\section{Queue-Detektion}


Um die Billard-Kugeln mit dem realen Queue kollidieren zu lassen, muss der Queue in dem von der Kamera aufgenommenen Bild erkannt und korrekt modelliert werden. 






\subsection{Segmentierung}
Beschreibung des Eingabebildes

Farbraumtransformation HSV

Thresholding im HSV Farbraum um Binärbild zu gewinnen 

Morphologisches Opening (Ausreißerentfernung) und Closing (Schließen der Lücken)


\subsection{Erkennung des Kollisionspunktes}
Binärbild $\textbf{B} \in \{0, 1\}^{n\times m}$ Ausgabe der Segmentierung

Da der Queue 

Hauptkomponentenanalyse:

Koordinaten in $\textbf{B}$ der als Queue klassifizierten Pixel als Zeilenvektoren:

$X = \{(x, y) \mid \textbf{B}_{x,y} = 1, x \in [1,n], y \in [1,m]\}$, $k = |\textbf{B}|$

Sei $\textbf{X}$ die $k\times 2$ Matrix der $k$ Zeilenvektoren aus $X$

Berechnung des Mittelwertes für beide Dimensionen:

$\textbf{u} = (u_1, u_2)$ mit $u_j = \frac{1}{k}\sum_{i=1}^{k}\textbf{X}_{i, j}, j \in \{1,2\}$

Normalisierung mittels des Mittelwertes:

$\textbf{A} = \textbf{B} - \textbf{h}\textbf{u}^{T}$ mit dem $k\times 1$ Spaltenvektor $\textbf{h}, h_i = 1$ für $i = 1,\dots,k$

Bestimmung der $2\times2$ Kovarianzmatrix:

$\textbf{C} = \frac{1}{k-1} {\textbf{B}}^{T}  \textbf{B}$









\subsection{Bewegungsinterpolation}