\chapter{Diskussion}
In diesem Kapitel soll die entwickelte Lösung kritisch betrachtet werden. Dabei soll gerade auf die Schwächen der Lösung eingangen und Verbesserungsvorschläge für diese dargestellt werden.



\section{Verbesserungsvorschläge}

\subsubsection{Rendering und Physik}

\subsubsection{Kamerakalibrierung}

\subsubsection{Queue-Detektion}
Die vorgestellte Erkennung des Queues basiert stark auf der Farbe des Queues. 
Da andere Gegenstände im Kamerabild eventuell ebenfalls einen dunklen Farbton können, wäre die Erkennung in diesem Fall sehr fehleranfällig. 
Um die Erkennung robuster gegenüber diesen Störfaktoren zu machen, könnte die Erkennung des Queues auf weiteren Merkmalen, wie zum Beispiel Markern, basieren.

Weiterhin limitiert die Wiederholfrequenz der Kamera von 25 Bildern die Sekunde stark die Präzision der Steuerung. 
Durch die hohe Belichtungszeit ergibt sich eine starke Bewegungsunschärfe, gerade bei schnellen Bewegungen. 
Diese wirkt sich negativ auf die genaue Erkennung der Queue Spitze aus.
Zusätzlich sorgt die Wiederholfrequenz dafür, dass das Spiel selbst auch nur mit 25 Bildern pro Sekunde dargestellt wird. 
Anders als das erste Problem könnte dies jedoch z.B. durch ein Double-Buffering der Eingabebilder behoben werden.

Schließlich lässt sich die Modellierung des Queues an sich verbesseren.
Die Kollision basiert in der vorgestellten Lösung lediglich auf einem Punkt, der Queue-Spitze.
In der Realität hat der Queue jedoch noch eine Ausdehnung in die Breite, die bei dieser Modellierung nicht berücksichtigt wird.