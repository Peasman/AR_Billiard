\chapter{Diskussion}
In diesem Kapitel soll die entwickelte Lösung kritisch betrachtet werden. Dabei soll gerade auf die Schwächen der Lösung eingangen und Verbesserungsvorschläge für diese dargestellt werden.



\section{Verbesserungsvorschläge}

\subsubsection{Rendering und Physik}
Die Optik des Spiels ist recht zweckmäßig: Das 2D sieht sorgt dafür, dass die Löcher des Spielfeldes und das  rein Rollen der Kugeln in diese schlecht aussieht. Man erkennt dadurch nicht richtig, ob man nun einen Punkt gemacht hat oder nicht, weil die Kugeln einfach verschwinden, wenn sie auf die Löcher treffen. Dies würde um einiges besser aussehen, wenn man das Spiel in 3D gemacht hätte, hätte aber auch dafür gesorgt, dass der Programmieraufwand für die Physik den Umfang des Projektes überstiegen hätte. \\
Bei der Phsyik gibt es diverse Fehler bei der Kollisionsberechnung. So kann es passieren, dass Kugeln eine unendliche (bzw. Maximal Große) Geschwindigkeit beim Kollidieren bekommen und einfach verschwinden. \\
Auch die Laufzeit der Kollisionsberechnung ist nicht gut. Diese ist Quadratisch nach der Anzahl der Kugeln. Dies ist jedoch nicht weiter schlimm, weil es nur begrenzt viele Kugeln geben kann und die Berechnung ansonsten trivial ist. Dies hätte man jedoch auch mit einem regulärem Gitter verbessern können, sodass die Kollision von Objekten untersucht wird, die tatsächlich in der Nähe voneinander sind.
 \subsubsection{Kamerakalibrierung}
Die Kalibrierung der Kamera, und damit die Erkennung des gegebenen Schachbrett-/Patternmusters bzw. deren Qualität, wird sehr durch den aktuellen Lichteinfluss, u.a. durch die Einstellungen des Beamers, auf das aufgenommene Kamerabild, beeinflusst.
Sofern die genutzte Kamera nicht exakt scharf eingestellt ist oder das Bild zu dunkel oder zu hell wird, hat dies als direkte Folge eine unsaubere bzw. schlechte bis nicht funktionierende Kamera-Kalibrierung. 
Da in unserem Projekt für die Erkennung des bekannten Schachbrettmusters eine bereits existierende Methode der OpenCV-Bibliothek genutzt wurde, sind die zu optimierenden Möglichkeiten sehr eingeschränkt.
Eine lösende Variante wäre es, ein vorgefertigtes Bild anzeigen zu lassen und erst bei einem aufgenommenem Foto dieses Bildes über die Kamera, welches entsprechende Merkmale aufweist (Helligkeit, Schärfe, Ausrichtung), die Weiterleitung an die Kalibrierung zu gestatten. Somit müsste der Benutzer die Kamera und den Beamer vor der Kalibrierung bereits so einstellen, dass diese Einstellungen eine erfolgreiche Kalibrierung zufolge haben.

\subsubsection{Queue-Detektion}
Die vorgestellte Erkennung des Queues basiert stark auf der Farbe des Queues. 
Da andere Gegenstände im Kamerabild eventuell ebenfalls einen dunklen Farbton können, wäre die Erkennung in diesem Fall sehr fehleranfällig. 
Um die Erkennung robuster gegenüber diesen Störfaktoren zu machen, könnte die Erkennung des Queues auf weiteren Merkmalen, wie zum Beispiel Markern, basieren.

Weiterhin limitiert die Wiederholfrequenz der Kamera von 25 Bildern die Sekunde stark die Präzision der Steuerung. 
Durch die hohe Belichtungszeit ergibt sich eine starke Bewegungsunschärfe, gerade bei schnellen Bewegungen. 
Diese wirkt sich negativ auf die genaue Erkennung der Queue Spitze aus.
Zusätzlich sorgt die Wiederholfrequenz dafür, dass das Spiel selbst auch nur mit 25 Bildern pro Sekunde dargestellt wird. 
Anders als das erste Problem könnte dies jedoch z.B. durch ein Double-Buffering der Eingabebilder behoben werden.

Schließlich lässt sich die Modellierung des Queues an sich verbesseren.
Die Kollision basiert in der vorgestellten Lösung lediglich auf einem Punkt, der Queue-Spitze.
In der Realität hat der Queue jedoch noch eine Ausdehnung in die Breite, die bei dieser Modellierung nicht berücksichtigt wird.

\subsubsection{Benutzerinteraktion}
Bei der Benutzerinteraktion wurde der Großteil der Interaktionsmöglichkeiten durch Buttons und Pop-ups realisiert.
Ein Teil dessen war allerdings auch das Anzeigen der Statistiken, worunter u.a. das Anzeigen des aktuellen Spielers oder der Kugelzuweisung fällt. Hierbei wurde die Variante gewählt, die Statistiken als Label auf dem Spielfeld anzeigen zu lassen.
Eine bessere Lösung wäre gewesen, diese in der Menübar oberhalb des Spielfeldes zu positionieren, um die Sicht des Benutzers auf das Spielfeld nicht zu beeinträchtigen. 
Das Problem lag in der Konnektivität der Physik- und der Anzeigeklasse, denn hier war es nicht einfach lösbar, Daten von der Physikklasse an die Anzeigeklasse weiterzuleiten.
Um dies effizient zu realisieren, hätte man dies direkt am Anfang in der allgemeinen Programmstruktur beachten müssen um einen einfachen Austausch von Daten nicht nur von Oberklassen an Unterklassen, sondern auch umgekehrt, zu gewährleisten.